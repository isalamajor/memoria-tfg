\section{Justificación de la Elección Tecnológica}
Para el desarrollo de esta aplicación se ha optado por Next.js como tecnología de frontend debido a sus ventajas en rendimiento, optimización para SEO y compatibilidad con React. Entre los aspectos más atractivos de esta opción se encuentra que cuenta con una plataforma de lanzamiento y hosting de aplicaciones gratuito, rápido y simple: Vercel; además de que la configuración inicial de un proyecto web se hace muy sencilla y existen numerosas librerías para facilitar el desarrollo web y numerosos recursos y componentes disponibles. 
En el backend, se ha elegido Node.js con Express por su flexibilidad y la posibilidad de utilizar JavaScript en todo el stack de desarrollo. En cuanto a la base de datos, se ha decidido emplear MongoDB por su modelo flexible y su capacidad para manejar estructuras de datos dinámicas y en formato JSON, el mismo en que se manejan las estructuras de información en Javascript. 
Finalmente, para el almacenamiento de archivos se utilizará Google Drive API, permitiendo a los usuarios gestionar sus documentos de manera segura y accesible desde cualquier dispositivo.
El desarrollo de aplicaciones web ofrece múltiples opciones tecnológicas, cada una con sus propias ventajas y desafíos. La elección de Next.js, Node.js con Express y MongoDB responde a la necesidad de construir una plataforma rápida, escalable y eficiente, capaz de gestionar un alto número de usuarios y permitir la colaboración en la corrección de textos. La integración con Google Drive añade un valor adicional al permitir a los usuarios almacenar y acceder a sus documentos de manera sencilla. Con esta arquitectura, la aplicación propuesta se encuentra bien posicionada para ofrecer una experiencia fluida y efectiva en el aprendizaje de idiomas mediante la escritura y la corrección colaborativa.
En cuanto a la metodología, se utilizará un modelo Kanban dado que es un enfoque ágil, simple y visual, ideal para proyectos en los que solo participa una persona. Permite gestionar el flujo de trabajo de manera rápida y flexible, sin necesidad de invertir demasiado tiempo en planificación detallada o documentación extensa. Su naturaleza visual facilita el seguimiento del progreso y la organización de tareas de forma intuitiva.
