\chapter{Negocio}

Hoy en día, existen varias plataformas y aplicaciones centradas en el aprendizaje colaborativo de idiomas, alguna de ellas, similar al proyecto aquí presentado, permiten la corrección de textos por parte de hablantes nativos. Estas herramientas varían en su enfoque, desde la corrección automática con inteligencia artificial hasta el intercambio lingüístico entre usuarios. Se expondrán varias de estas alternativas para estudiar en qué se diferenciará el proyecto aquí expuesto de estas y qué ideas, características y funcionalidades nuevas aporta.

\subsubsection{HiNative}
HiNative es una de las plataformas más conocidas en este ámbito, diseñada para resolver dudas sobre el uso de un idioma a través de preguntas y respuestas. Los usuarios pueden consultar sobre gramática, pronunciación y expresiones cotidianas, obteniendo respuestas de hablantes nativos. No obstante, HiNative no está enfocada en la corrección de textos largos, sino en consultas puntuales, lo que limita su utilidad para quienes buscan mejorar su escritura de manera estructurada. Además, al depender de la comunidad, la calidad y rapidez de las respuestas pueden variar considerablemente.
\begin{figure}[H]
    \centering
    \includegraphics[height=3cm]{imagenes/hinative.png}
    \caption{Logo de HiNative}
    \label{fig:hinative}
\end{figure}


\subsubsection{Lang-8}
Por otro lado, Lang-8 ofrece un sistema más orientado a la corrección de textos completos. Los usuarios publican textos en el idioma que están aprendiendo y reciben correcciones detalladas de hablantes nativos. Esta plataforma permite obtener retroalimentación más estructurada que HiNative, ya que el objetivo es mejorar la redacción en su conjunto. Sin embargo, el proyecto ha dejado de recibir actualizaciones y no cuenta con una integración eficiente en dispositivos móviles, lo que ahora limita su accesibilidad y usabilidad.

\begin{figure}[H]
    \centering
    \includegraphics[height=3cm]{imagenes/lang-8.jpg}
    \caption{Logo de Lang-8}
    \label{img:lang8}
\end{figure}
\subsubsection{Tandem}
Otra alternativa es Tandem, una aplicación centrada en el intercambio lingüístico en tiempo real. A través de mensajes de texto, llamadas de voz y videollamadas, los usuarios pueden practicar con hablantes nativos, incluyendo la opción de corregir mensajes dentro del chat. No obstante, Tandem prioriza la interacción oral y escrita breve, sin ofrecer herramientas específicas para la corrección detallada de textos largos ni la posibilidad de almacenarlos para un seguimiento a largo plazo.
\begin{figure}[H]
    \centering
    \includegraphics[height=3cm]{imagenes/tandem.png}
    \caption{Logo de Tandem}
    \label{fig:tandem}
\end{figure}

\subsubsection{Grammarly}
En el ámbito de la corrección automática, herramientas como Grammarly proporcionan análisis gramatical y sugerencias estilísticas mediante inteligencia artificial. Si bien son útiles para detectar errores básicos y mejorar la fluidez del texto, su funcionamiento se basa en reglas predefinidas y modelos de aprendizaje automático, por lo que no pueden ofrecer la misma precisión y contexto que un hablante nativo. Además, carecen de un enfoque social o de intercambio colaborativo entre usuarios.
\begin{figure}[H]
    \centering
    \includegraphics[height=3cm]{imagenes/grammarly.png}
    \caption{Logo de Grammarly}
    \label{fig:grammarly}
\end{figure}

\subsubsection{Duolingo}
Duolingo es una de las aplicaciones más populares para el aprendizaje de idiomas, conocida por su enfoque gamificado y accesible. A través de ejercicios interactivos, los usuarios pueden mejorar su vocabulario, gramática y comprensión escrita y oral de un idioma. Cuenta con un sistema de recompensas que fomenta la constancia en el estudio. Sin embargo, aunque Duolingo ofrece ejercicios de traducción y escritura, su metodología se centra en frases cortas y respuestas predefinidas, sin permitir la redacción libre ni la corrección detallada de textos extensos.
\begin{figure}[H]
    \centering
    \includegraphics[height=3cm]{imagenes/duolingo.png}
    \caption{Logo de Duolingo}
    \label{fig:duolingo}
\end{figure}

\subsubsection{Conclusión}
A pesar de existir varias opciones similares, ninguna ofrece una solución que combine la escritura libre con la interacción social de manera sencilla y atractiva. La aplicación propuesta en este documento se diferencia en que permite a los usuarios intercambiar textos personales con amigos o nuevos contactos, eligiendo la temática de sus escritos, como si fueran entradas de un diario. Además, incorpora un sistema de almacenamiento de textos, permitiendo a los usuarios llevar un registro de su evolución en la escritura del idioma que están aprendiendo.
A diferencia de estas plataformas, la aplicación propuesta busca ofrecer un espacio donde los usuarios puedan expresarse con mayor libertad, intercambiar textos personales con otros estudiantes y recibir correcciones estructuradas en un entorno colaborativo.
Uno de los objetivos principales de este proyecto es ofrecer una interfaz intuitiva y visualmente atractiva, diseñada específicamente para facilitar la corrección de textos de manera sencilla y fluida. A diferencia de otras plataformas que presentan sistemas de corrección poco intuitivos o sobrecargados de opciones, esta aplicación prioriza una experiencia de usuario agradable, donde las correcciones pueden realizarse de manera clara y organizada. La combinación de un diseño amigable con la posibilidad de conocer nuevas personas a través del intercambio de textos hace que esta plataforma no solo sea una herramienta educativa, sino también un espacio de interacción y aprendizaje dinámico.

